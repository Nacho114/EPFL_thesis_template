%\begingroup
%\let\cleardoublepage\clearpage


% English abstract
\cleardoublepage
\chapter*{Abstract}
\markboth{Abstract}{Abstract}
% \addcontentsline{toc}{chapter}{Abstract} % adds an entry to the table of contents
% put your text here
% \lipsum[1-2]
\vspace{0.2\textheight}

Understanding the relationships among objects is in some sense half of the study of science; the other 
half being the definition and discovery these objects. Causality tries to understand the question of 
relation; if we are given random variables $X$ and $Y$, under what conditions can we infer that 
one causes the other, and if so, how? 


If we are able to intervene on a system, then the problem is more straightforward; this is essentially
how we learn growing up. The second and more challenging scenario is the observational setting: given 
only observations of $X$ and $Y$, what can we infer? We will tackle the bivariate setting, which has 
seen various methods proposed over the years. We will review the most common family of such methods: 
the additive noise models (ANM), as well as some more recent ones such as the causal generative neural 
network (CGNN). 

We will propose a new\footnote{To the knowledge of the author no such method has been published 
before, although given it's simplicity we suspect that its not the first of its kind.} 
type of inference method which exploits the i.i.d 
noise assumption; the idea is to split the data in different intervals, and then to regress each 
interval separately. In the causal direction, one would expect that the residuals of each 
interval will be more homogenous --- in the ANM settings, asymmetries introduced
by inversion will break the i.i.d assumption. We will conclude by 
showing that this intuition is indeed correct, by proving that the method is consistent --- 
assuming that causal discovery is indeed possible and that the additive noise is i.i.d. We will
also propose a second method, which while also consistent, requires knowledge of the 
noise distribution, which is a poor assumption. We will show that our (first) method is competitive with 
other methods in the popular benchmarks such as the Tübingen cause and effect dataset. 



% For other languages:


% % German abstract
% \begin{otherlanguage}{german}
% \cleardoublepage
% \chapter*{Zusammenfassung}
% \markboth{Zusammenfassung}{Zusammenfassung}
% % put your text here
% \lipsum[1-2]
% \end{otherlanguage}


% % French abstract
% \begin{otherlanguage}{french}
% \cleardoublepage
% \chapter*{Résumé}
% \markboth{Résumé}{Résumé}
% % put your text here
% \lipsum[1-2]
% \end{otherlanguage}


%\endgroup			
%\vfill

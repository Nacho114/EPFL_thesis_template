%\begingroup
%\let\cleardoublepage\clearpage


% English abstract
\cleardoublepage
\chapter*{Abstract}
\markboth{Abstract}{Abstract}
% \addcontentsline{toc}{chapter}{Abstract} % adds an entry to the table of contents
% put your text here
% \lipsum[1-2]

Understanding the relationships among objects is in some sense half of the study of science; the other 
half is to define and discover these objects. The question of causality is more straightforward if we 
are able to intervene on a system -- this is indeed how we learn as we grow up. What happens when we 
are unable to perform any interventions and only make observations? When are we able to make any sort 
of causal inference? If so, how? We will explore and answer some of the questions for the most basic 
setting, that is, the bivariate setting with two random variables $X$ and $Y$. When can we deduce that 
$X$ causes $Y$? We will review classical methods such as the additive noise model (ANM), and more recent 
ones such as the causal generative neural networks (CGNN). We will also review some theory that tells us 
when causal is possible in the bivariate setting under some assumptions about the causal model (how $Y$
depends on $X$). We will conclude by showing a new type of method -- inspired by the ANM type scoring 
methodology. Instead of testing for independence between residuals and inputs we will test the consistency 
of the residual. 


% For other languages:


% % German abstract
% \begin{otherlanguage}{german}
% \cleardoublepage
% \chapter*{Zusammenfassung}
% \markboth{Zusammenfassung}{Zusammenfassung}
% % put your text here
% \lipsum[1-2]
% \end{otherlanguage}


% % French abstract
% \begin{otherlanguage}{french}
% \cleardoublepage
% \chapter*{Résumé}
% \markboth{Résumé}{Résumé}
% % put your text here
% \lipsum[1-2]
% \end{otherlanguage}


%\endgroup			
%\vfill

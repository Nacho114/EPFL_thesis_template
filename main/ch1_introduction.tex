% \cleardoublepage
% \chapter*{Introduction}
% \markboth{Introduction}{Introduction}
% \addcontentsline{toc}{chapter}{Introduction}

\chapter{Introduction}

\section{Problem and Motivation}

Suppose we are given samples of data say $X$ and $Y$, s.t.

\begin{align*}
    X = x_1, ..., x_n  \\
    Y = y_1, ..., y_n 
\end{align*}

For example, we may be measuring the blood pressure and heart rate of Alice at time $k$, 
say $x_k$ and $y_k$ respectively. Further, suppose we are unaware of her context, for example,
Bob hacked into Alice's apple watch and so can only read $X$ and $Y$ -- but he has no idea of 
anything she might be up to.

Bob then observes the following trend:

\begin{figure}[H]
    \centering
    \includegraphics[width=.45\textwidth]{correlated_hr.png}
    \caption{Heart rate (HR) and Blood pressure (BP) of Alice.}
\end{figure}

Bob, having studied data science, is well aware of the fallacy of the law of small numbers\footnote{
    The law of small numbers is the error of concluding too much from too few data. 
}. He therefore checks again the data the next day at a slightly different time. He again observes 
a similar trend, and is now more confident in the existance of a causal relation and -- having neglected biology as 
being beneath him -- makes the conjecture that either blood pressure causes heart rate, or 
perhaps the other way around. 

Given this strong correlation, Bob asserts that he may either model $X$ as a function of $Y$ or the other way 
around. He proceeds to find some $f$ s.t. $f(X) \approx Y$. The next day, to his dismay, he notices that his
model has terrible performance when evaluated on new data. He then proceeds to see what is going on, and 
observes the following:

\begin{figure}[H]
    \centering
    \includegraphics[width=.45\textwidth]{uncorrelated_hr.png}
    \caption{Heart rate (HR) and Blood pressure (BP) of Alice.}
\end{figure}

As it turns out, in the last few days, Alice was working hard on finishing her thesis and the deadline had been 
the previous day. But how, Bob wondered, could this have changed the relationship between BP and HR? 
Finally, admitting to himself that machine learning alone is not enough to understand the world; 
Bob spends some time learning about the heart. It turns out, that fear triggers a "flight or fight"
response that increases both the heart rate and blood pressure; Interestingly your heart rate and blood pressure 
won’t always rise and fall in sync.

So what did Bob learn\footnote{
    Note that heart rate and blood pressure are intimately linked, and the story between them is more complicated.
    The plots were randomly generated using a gaussian process, however they do resemble some real examples that 
    can be found in google images.
}?

\begin{enumerate}
    \item When we train a model with some data, when we use it on some newly aquired data, we might
    face a \textbf{covariate-shift} -- that is, the distribution might change due to the context changing.
    \item When we see correlation it might be spurious due to a \textbf{confounder} -- fear was the \textbf{confounder} 
    of the heart rate and blood pressure.
    \item His degree in Data Science is worth less than he thought; \textbf{machine learning is in fact not 
    a panacea}, contrary to common culture. However, applied with domain knowledge and causal reasoning
    it may very useful. 
\end{enumerate}

If Bob was able to incorporate these notions into his machine learning models, then it might have been more
robust to the covariate-shift. To give a more concrete example, there is a "neural net tank urban legend"\footnote{
    More about this story here: https://www.gwern.net/Tanks.
}
, where a neural network accuratly predicts if there is a tank or not in an image, but it turns out it uses the 
weather as a predictor. From this it is clear that the model will preform badly under covariate shift, and indeed
it makes the case that incorporating causality to a model should make it more robust as
\cite{scholkopf2019causality} argues. Note that this is in effect the issue with generalisation in machine learning:
how can we ensure that we learn \textit{meaningful} representations (features about the tanks) rather than just
correlations (the weather) useful for train accuracy. 

As for confounders, it is impossible to say anything in general\footnote{
    For most of the 20th century, a huge debate took place to determine the question of whether or not 
    smoking caused cancer. A clever argument against a causal relation was that there existed a gene that 
    made a person both want to smoke and more prone to cancer; even the father of modern statistics
    himself thought this explanation more plausible (For a good read on how science is and was used 
    for wrong see the excellent book \cite{NaomiMerchants}).
    
}. We must therefore specify a causal model, and
then see what gurantees we can give under what assumptions. Even in the absence of confounders it is highly 
non trivial to determine causality.

As this simple example illustrates, causality is related to many interesting questions; perhaps, one of the most 
simple questions we can ask -- and the one that we will explore -- is, given that either X causes Y, or Y causes X
(we assume no confounders) then, when 
can we predict the direction of causality? If yes, how? 

In the figure bellow (figure \ref{fig:simple_bivariate_example}) can you tell if $X$ causes $Y$? Or perhaps
it is the other way around? The right answer is that $X$ causes $Y$, and we will show algorithms that 
can accuretly predict causality in such settings with as few as 75 samples. 

\begin{figure}[H]
    \centering
    \includegraphics[width=1\textwidth]{bivariate_causal_example.png}
    \caption{75 samples of data $X$, $Y$.  The samples are generated independently as follows:
    $y_i = f(x_i) + n_i$ where $x_i$ is drawn from an expoential distribution and $n_i$ is drawn 
    independently from a gaussian one and $f(x) = 10 \tanh(x) + 4\sin(x) + x + x^2$}
    \label{fig:simple_bivariate_example}
\end{figure}

\section{Causality}

We have seen in the introduction that determining and understanding causality may indeed be useful; 
but what exactly is causality? When we say $X$ causes $Y$ i.e. $X \rightarrow Y$, we have an intuitive
idea of what it means; but how should we formalise it?

Suppose that we are given two random variable $X$, $Y$ with joint distribution $p_{x, y}$. Intuitively we 
would say that $X \rightarrow Y$, if we intervene in $X$ and then see an effect $Y$. In particular
we will denote $\operatorname{do}(x)$ -- short for $\operatorname{do}(X = x)$ -- as an intervention
that forces the variable $X$ to have the value $x$, and leaves the rest of the system untouched. 
Following the convention inspired by \cite{pearl2000causality}, 
we define the resulting distribution as $p_{Y|do(x)}$.

In supervised learning the goal is often to estimate $p_{y|x}$; while it is tempting to say that this is the 
same as $p_{y|do(x)}$, they may in fact be very different. Suppose $x$ is 1 if a person smokes and 0 otherwise, 
similarly let $y$ be 1 if a person has cancer and 0 otherwise. If we then ran immoral 
\textbf{randmoized trials}\footnote{The power of randmoized trials is that they average out potential confounder 
-- e.g. a gene that causes both an inclination to smoke and cancer. This is also knowm as A/B testing and is 
essential to determine if a new medicine has the desired effect and that a new UI might maximise user addicton 
to an app.} and forced a subset of the population at random to smoke, then we could estimate\footnote{
Note that to estimate $p_{y|x}$ we simply need to sample people at random to determine $p_{y, x}$, we can 
then use bayes' formula to determine $p_{y|x} = \frac{p_{y, x}}{p_x}$
} $p_{y|do(x)}$. 
If as some people believed, there existed a gene that made people both smoke and prone to cancer, then 
$p_{y|do(x)} = p_{y|x}$; however, 

This motivates the following definition:

\begin{definition}
    We say that $X$ \textbf{causes} $Y$ if $p_{y|do(x)} \neq p_{y|do(x^\prime)}$ for some
    $x, x^\prime$
\end{definition}

Note that causal relations can also be \textit{cyclic}, i.e. $X$ causes $Y$ which in turn causes 
$X$ ad infinitum. While this deserves consideration as many systems have feedback loops we will
for simplicty not look at such setting.

If we have the random variables $X$, $Y$, and $Z$, then depending on how they are causaly linked, then 
we will have different relationships between the marginals and the Pearl's "do" conditional. For example,
if $X \rightarrow Z \rightarrow Y$, then we will have that $p_X = p_{X|do(y)}$, but $p_Y \neq p_{Y|do(x)}$. 

Then, assuming that we get both interventional and observational data it suffices to check to which of the 6 
possible structure the data corresponds to (at least in theory). However, if we are interested in using only 
observational data, then we will need to make conessions; in particular, we will make some assumptions about 
the causal structure, and then if we can show that we can infer the causal structure in such a setting, we 
shall call it \textbf{identifiable}.

If we have a 3 or more varuables, one can use conditional independence tests to check whether two random 
variables are directly linked (to check for direct causality -- the case when there is an edge from X to Y).
One can extend the notion of conditional independence to consider sets of edges instead of single edges, 
see for example graphical models.

In some sense the 2d case is hard because we cannot use conditional independence tests. 

The underlying structure behind a causal model is what is known as Strucutual Equation Models (SEM). Essentially 
it is a model specification; and the key insight is that it should not be reversible. 


\section{Proposed Methods}

We propose two methods to deal with the ANM -- one less practical, but with a nice theoretical
analysis; and another more practical, but perhaps with a less pleasing analysis. However both
are based on a first principle approach with known asymptotics in mind.

We note that in the analysis / procedure we split the data in 2, first to train the model, and
then to perform the score computation - Need to expand on this

Will be easier to do once they are finished.


\section{Outline}

The thesis is divided in two parts, 

background etc blup blue die blah blu

Do once finished, but main idea

Theory and background material

Methods

Experiments
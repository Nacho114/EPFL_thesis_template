\cleardoublepage
\chapter*{Introduction}
\markboth{Introduction}{Introduction}
\addcontentsline{toc}{chapter}{Introduction}

The endevour of science is in some sense the uncovering of causal structures: does the mass
of an object influence its acceleration in free fall? What genomes influence height (etc)? 
The ability to do experiments in Physics is what has allowed to confirm or uncover relations
among objects; indeed, this is also how we learn best, by tweaking a system and having a 
direct feedback which allows us to evaluate our mental models. In the realm of causality, 
such a setting is what is known as the "Causal intervention framework" (need to check). 
Give quickly example about smoking, and illustrate how it would be harder without interventions.



An excellent question about a multivariable causal system is to ask, "what is the minimum number
of interventions one has to do to achieve some alpha-confidence about the causal effect"


How can we tackle causality?

In the absence of noise, and the process is bijective, then it is impossible to distinghuis, 
if however, ...

Shannon answered the question: given the most simple communication system: "How reliably can 
we communicate given a certain noise level"

In some sense what we would like to answer is, given a certain noise level, how reliably can we 
predict the causal relation. 

Some points:

1. In causality we use noise, whereas in virtually all other domains
 such as communication theory the aim is combat noise.

 Interesingly yet again, the Gaussian case ends up being a difficuly case. For instance, 
 the motivation to look at the AGN additive gaussian noice channel is that the gaussian is 
 the most difficult distribution in the entropic sense; but so it is as well in the 
 bianry case setting due to:

 thm.


A non-numbered chapter\dots


Talk about SNR, role with shannon, and how it affects prediction in a reverse way here! Cite shanon!


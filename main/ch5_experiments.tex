\chapter{Experiments}


\section{Benchmark}

We benchmark on five bivariate cause-effect datasets\footnote{ The TCEP dataset 
can be found \href{https://webdav.tuebingen.mpg.de/cause-effect/}{\color{blue}{here}} and 
all the other datasets can be found 
\href{https://dataverse.harvard.edu/dataset.xhtml?persistentId=doi:10.7910/DVN/3757KX}{\color{blue}{here}}
}
, covering a wide range of associations:

% \begin{enumerate}[itemsep=-1.8em, topsep=0pt]
\begin{enumerate}[noitemsep, topsep=0pt]

    \item \textbf{Cha} (300 cause-effect pairs) pairs from the challenge of \cite{chalearn}
    \item \textbf{Net} (300 cause-effect pairs) artificial cause-effect pairs generated using 
            random distributions as causes, and neural networks as causal mechanisms
    \item \textbf{Gauss} (300 cause-effect pairs) generated by \cite{Mooij2016jmlr}, using random 
      mixtures of Gaussians as causes, and Gaussian process priors as causal mechanisms.
    \item \textbf{Multi} (300 cause-effect pairs) built with random linear and 
    polynomial causal mechanisms (by \cite{goudet2017causal}). In this dataset, 
    additive or multiplicative noise, is applied before or after the causal mechanism.
    \item \textbf{TCEP} (108 cause-effect pairs)\footnote{Note that 6 of these pairs are not bivariate.} 
     is the Tübingen Cause Effect Pair data set which consists of various 
    domains such as climatology, finance, and medicine (\cite{Mooij2016jmlr}).

\end{enumerate}

Cite competitors. (as done in GCNN)

The first is via the Area Under the Precision Recall curve, and the second 
only checks at accruacy. 

\begin{table}[H]
    \centering


    \begin{tabular}{lccccc}
        \hline method & Cha & Net & Gauss & Multi & TCEP \\
        \hline Best fit & 56.4 & 77.6 & 36.3 & 55.4 & 58.4 (44.9) \\
        LiNGAM & 54.3 & 43.7 & 66.5 & 59.3 & 39.7 (44.3) \\
        CDS & 55.4 & 89.5 & 84.3 & 37.2 & 59.8 (65.5) \\
        IGCI & 54.4 & 54.7 & 33.2 & 80.7 & 60.7 (62.6) \\
        ANM & 66.3 & 85.1 & 88.9 & 35.5 & 53.7 (59.5) \\
        PNL & 73.1 & 75.5 & 83.0 & 49.0 & 68.1 (66.2) \\
        Jarfo & 79.5 & 92.7 & 85.3 & 94.6 & 54.5 (59.5) \\
        GPI & 67.4 & 88.4 & 89.1 & 65.8 & 66.4 (62.6) \\
        CGNN $\left(\widehat{\mathrm{MMD}}_{k}\right)$ & 73.6 & 89.6 & 82.9 & 96.6 & 79.8 (74.4) \\
        CGNN $\left(\widehat{\mathrm{MMD}}_{k}^{m}\right)$ & 76.5 & 87.0 & 88.3 & 94.2 & 76.9 (72.7) \\
        \hline TwinTest & 66.3 & 81.9 & 85.1 & 39.8 & 77.0 (82.0) \\
        \hline
    \end{tabular}

    \caption{Cause-effect relations: Area Under the Precision Recall curve
     on 5 benchmarks for the cause-effect experiments (weighted accuracy 
     in parenthesis for TCEP)}   
    \label{tab:AUPR}
\end{table}

Table \ref{tab:AUPR} is taken from \cite{goudet2017causal}

\begin{table}[H]
    \centering

    \begin{tabular}{lcc}
        \hline Model & TCEP & TCEP with 75 samples \\
        \hline BCI & 0.64 & 0.60 \\
        ANM-HSIC & 0.63 & 0.54 \\
        ANM-MML & 0.58 & 0.56 \\
        IGCI & 0.66 & 0.62 \\
        CGNN & 0.70 & 0.69 \\
        \hline TwinTest & 62.4 & TODO \\
        \hline
    \end{tabular}
    \caption{Accuracy for TCEP Benchmark} 
    \label{tab:acc}
\end{table}

Table \ref{tab:acc} is taken from \cite{kurthen2018bayesian}
